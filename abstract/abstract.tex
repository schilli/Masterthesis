\documentclass[english, a4paper, 12pt, notitlepage, final]{article}
\usepackage[british]{babel}
\usepackage{graphicx,hyperref,url,color,cite}
\usepackage[latin1]{inputenc}

\title{\textbf{Masterthesis} \\
Elucidating the structure of a fusion protein of a lipase and a citrate binding domain by Molecular Dynamics simulation
}

\author{
Oliver Schillinger \\
Forschungszentrum J\"ulich \\ Institute of Complex Systems 6 - Structural Biochemistry \\ Multiscale Modelling Group \\
\textit{Supervisor}: Jun.-Prof. Dr. Birgit Strodel
}

\date{4th November 2013}
 

\begin{document}

%\maketitle

\begin{center}
{\huge \textbf{Masterthesis} \\}
\vspace {1cm}
{\large Elucidating the structure of a fusion protein of a lipase and a citrate binding domain by Molecular Dynamics simulation \\
}

\vspace{1cm}
Oliver Schillinger \\
Forschungszentrum J\"ulich \\ Institute of Complex Systems 6 - Structural Biochemistry \\ Multiscale Modelling Group \\
\textit{Supervisor}: Jun.-Prof. Dr. Birgit Strodel

4th November 2013

\end{center}
 


\begin{abstract}

A fusion protein of the periplasmic domain of sensor histidine kinase CitA from \textit{Klebsiella Pneumoniae} and the lipase LipA from \textit{Bacillus Subtilis} is investigated by means of molecular dynamics simulation.
The research was motivated by experiments indicating that CitA's natural ligand citrate down-regulates lipase activity upon binding.
Both proteins were first studied independently.
It was found that the citrate binding pocket was stable in the citrate bound crystal structure when studied in a 100 ns molecular dynamics simulation after careful equilibration.
The unfused, solo lipase was also very stable during a simulation of the same length, especially the binding pocket and the catalytic residues maintained their relative conformations.
As no crystal structure of the fusion protein is known, several possible structures have been generated \emph{in silico} by Basin Hopping, the most promising four of which have been investigated in depth by molecular dynamics simulations.
The fusion protein showed dynamics that differ significantly from those of the isolated proteins.
The presence of citrate in the receptor binding pocket did not stabilize the structure to the same extend compared to the isolated CitA.
The diversity of the obtained results indicate a strong dependence of the fusion protein dynamics on the initial conformation.
Therefore, experimental data is needed to discriminate between the possible fusion protein conformations on an empirical basis.

\end{abstract} 

\end{document}
