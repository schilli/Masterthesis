% Talk at High Throughput MD meeting in Barcelona
% Multiscale Modelling Group of Jun.-Prof. Birgit Strodel
% Given on November 08, 2013 at PPRB, Barcelona
%
% Oliver Schillinger

\documentclass[english]{beamer}

\usetheme{Juelich}
\usepackage[scaled]{helvet}

\usepackage[british]{babel}
\usepackage{graphicx,hyperref,url,color}
\usepackage{amsmath}
\usepackage[latin1]{inputenc}
\usepackage[T1]{fontenc} 
\usepackage{setspace}

%\usepackage[perpage,para,symbol]{footmisc}
%\renewcommand\footnotelayout{\tiny}
\usepackage{biblatex}
\bibliography{masterthesis}

\usepackage{ulem}

\setbeamercovered{transparent}
\setbeamertemplate{slide counter}[showall][]

\hypersetup{
    %bookmarks=false,        % show bookmarks bar?
    unicode=false,          % non-Latin characters in Acrobat’s bookmarks
    pdftoolbar=true,        % show Acrobat’s toolbar?
    pdfmenubar=true,        % show Acrobat’s menu?
    pdffitwindow=false,     % window fit to page when opened
    pdfstartview={FitH},    % fits the width of the page to the window
    pdftitle={Talk Masterthesis},    % title
    pdfauthor={Oliver Schillinger},     % author
    pdfsubject={Structure of Lipase-CitAP complex},   % subject of the document
    pdfcreator={Oliver Schillinger},   % creator of the document
    pdfproducer={Oliver Schillinger}, % producer of the document
    pdfkeywords={Lipase} {Citrate} {CitAP}, % list of keywords
    pdfnewwindow=true,      % links in new window
    colorlinks=true,        % false: boxed links; true: colored links
    linkcolor=black,          % color of internal links
    citecolor=green,        % color of links to bibliography
    filecolor=magenta,      % color of file links
    urlcolor=blue           % color of external links
}

% ============================================================================ %
%
% Multi column example
%
%\begin{frame}
%    \frametitle{Some title}
%    \framesubtitle{Some subtitle}
%
%    \begin{columns}[t]
%        \column{.5\linewidth}
%            Some text
%
%        \pause
%
%        \column{.5\linewidth}
%            Some more test
%
%    \end{columns}
%
%\end{frame}
%
% ============================================================================ % 

% Titlepage
\title[Masterthesis]{Masterthesis}
\subtitle[Structure]{Structure of Lipase-CitAP Fusion Protein\\
\textit{\small Supervisor: Jun.-Prof. Dr. Birgit Strodel}}
\author[o.schillinger@fz-juelich.de]{Oliver Schillinger}
\institute{ICS-6 | Multiscale Modelling Group}
\date{8th November 2013} 

% ============================================================================ %

% begin document
\begin{document}

\maketitle

\begin{frame}
    \frametitle{Outline}
    \tableofcontents
\end{frame}

% ============================================================================ %

\section{Proteins}

\pagenumbering{arabic}

\begin{frame}
    \frametitle{Proteins}
    \framesubtitle{BsLA}

    \begin{columns}[t]
        \column{.6\linewidth}
        \begin{itemize}
            \item \textit{Bacillus subtilis} Lipase A
%            \item Minimal $\alpha/\beta$ hydrolase fold enzyme
            \item Structures down to 1.3 \r{A} resolution
            \item Catalyses hydrolysis and synthesis of \textbf{triacylgliycerols}
            \item Diverse substrate specificity
            \item Used in industry for
                \begin{itemize}
                    \item Resolution of racemic mixtures
                    \item Synthesis of esters
                    \item Additive laundry detergent
                \end{itemize}
        \end{itemize} 

        \column{.4\linewidth}
        \begin{figure}
            \includegraphics[width=.9\linewidth]{figures/Lipase.png}
        \end{figure}     
        \tiny Bacillus subtilis lipase A with covalently bound Rc-IPG-phosphonate inhibitor (PDB: 1R4Z)

    \end{columns} 

\end{frame}

% ============================================================================ %

%\begin{frame}
%    \frametitle{Proteins}
%    \framesubtitle{BsLA}
%
%    \begin{columns}[t]
%        \column{.6\linewidth}
%        \begin{itemize}
%            \item 181 amino acids
%            \item Much smaller than lipases from other organisms
%            \item Active site not covered by lid but solvent exposed -- \textbf{no interfacial activation} (activation due to lid opening in presence of lipid aggregates)
%        \end{itemize} 
%
%        \column{.4\linewidth}
%        \begin{figure}
%            \includegraphics[width=\linewidth]{figures/Lipase_Lid.png}
%        \end{figure}     
%        \tiny BsLa lacks a lid covering the active site (Lid taken from P. Aeruginosa lipase, PDB: 1EX9)
%
%    \end{columns} 
%
%\end{frame} 

% ============================================================================ %

\begin{frame}
    \frametitle{Proteins}
    \framesubtitle{BsLA}

    \begin{columns}[t]
        \column{.5\linewidth}
        \begin{figure}
            \includegraphics[width=1.0\textwidth]{figures/BSLA_pocket/BSLA_pocket_cartoon.pdf}
        \end{figure}      

        \column{.5\linewidth}
        \begin{figure}
            \includegraphics[width=1.0\textwidth]{figures/BSLA_reaction.png}
        \end{figure}     

    \end{columns} 

    \centering
    His156 -- Asp133 distance related to lipase activity

\end{frame}  

% ============================================================================ %

\begin{frame}
    \frametitle{Proteins}
    \framesubtitle{CitAP}

    \begin{itemize}
        \item Periplasmic domain of a two component system of a sensor and a response regulator
        \item CitA: Sensor Histidine Kinase
        \item CitB: is the corresponding response regulator
        \item \textit{Klebsiella pneumoniae} two component system is essential for the induction of citrate fermentation genes in the presence of citrate
    \end{itemize}
\end{frame} 

% ============================================================================ %

\begin{frame}
    \frametitle{Proteins}
    \framesubtitle{CitAP Signal Transduction Mechanism}
    \begin{figure}
        \includegraphics[width=.9\linewidth]{figures/CitA_mechanism.png}
    \end{figure}      

    \tiny
    \fullcite{CitA_2J80}

%    M. Sevvana, et. al,
%    \href{http://www.sciencedirect.com/science/article/pii/S0022283608000466}
%    {A Ligand--Induced Switch in the Periplasmic Domain of Sensor Histidine Kinase CitA},
%    \textit{J. Mol. Biol.}, 377, 512--523, 2008

\end{frame}  
 
% ============================================================================ %

%\begin{frame}
%    \frametitle{Proteins}
%    \framesubtitle{CitAP active site opening}
%    \begin{figure}
%        \includegraphics[width=.8\linewidth]{figures/CitA_dimer.pdf}
%    \end{figure}      
%
%\end{frame}   

% ============================================================================ %

\begin{frame}
    \frametitle{Proteins}
    \framesubtitle{CitAP Binding Pocket}

    \begin{figure}
        \includegraphics[width=.7\linewidth]{figures/CitA_pocket2.pdf}
    \end{figure}     
\end{frame}     

% ============================================================================ %

\begin{frame}
    \frametitle{Proteins}
    \framesubtitle{CitAP Citrate Interactions}

    \begin{figure}
        \includegraphics[width=0.9\textwidth]{figures/citrate_interactions/citrate_interactions.pdf}
    \end{figure}     
\end{frame}      

% ============================================================================ %

\begin{frame}
    \frametitle{Proteins}
    \framesubtitle{Complex}

    \setbeamercovered{invisible}

    \begin{block}{Why make a complex?}
        If citrate could trigger lipase activity we had a switchable detergent!
    \end{block}

    \pause

    \begin{columns}[t]
        \column{.88\linewidth} 
        \vspace{-3ex}
        \begin{figure}
            \includegraphics[width=0.7\textwidth]{figures/BSLA_activity/BSLA_activity.png}
        \end{figure}       
        \vfill

        \column{.12\linewidth} 
        \tiny
        Jaeger et al.
    \end{columns}

\end{frame}  

% ============================================================================ %

\begin{frame}
    \frametitle{Proteins}
    \framesubtitle{Complex}

    \begin{figure}
        \includegraphics[width=.9\linewidth]{figures/complex/complex_folding.pdf}
    \end{figure}       

\end{frame}   

% ============================================================================ %

\section{Methods}

\begin{frame}
    \frametitle{Methods}
    \framesubtitle{MD} 

    \begin{itemize}
        \item GROMACS
        \item \textbf{\texttt{amber99sb-ildn-nmr}} force field
        \item Citrate paremeterization with GAFF
        \item 10 ns position restrained equilibration
        \item Gradually decreasing restraining force constant
        \item 100 ns production runs
        \item PBC, NPT, PME
        \item Mixed hardware: Clusters (up to 180 cores), GPUs (up to 4)
    \end{itemize}

\end{frame}    

% ============================================================================ %

\begin{frame}
    \frametitle{Methods}
    \framesubtitle{GMIN: Basin Hopping with Global Optimization}

    \begin{figure}
        \includegraphics[width=0.9\textwidth]{figures/GMIN/GMIN.pdf}
    \end{figure}        

\end{frame}    



% ============================================================================ %

\section{Results}

\begin{frame}
    \frametitle{Results}
    \framesubtitle{Solo CitAP Binding Pocket}

    \vspace{1.0\topmargin}

    \begin{figure}
        \hspace{0.2\textwidth}
        \includegraphics[width=.4\linewidth]{figures/CitA_pocket2.pdf}
    \end{figure}     

    \vspace{-1.2cm}

    \begin{columns}[t]
        \column{.5\linewidth} 
        \begin{figure}
            \includegraphics[width=1.0\textwidth]{figures/CitAP_opening/CitAP_dist_free.pdf}
        \end{figure}       

        \column{.5\linewidth} 
        \begin{figure}
            \includegraphics[width=1.0\textwidth]{figures/CitAP_opening/CitAP_dist_bound.pdf}
        \end{figure}        

    \end{columns} 


\end{frame}    

% ============================================================================ %

\begin{frame}
    \frametitle{Results}
    \framesubtitle{Solo BSLA MD} 

    \vspace{0.10\topmargin}

    \begin{figure}
        \begin{minipage}[t]{0.45\linewidth}
            \centering
            \includegraphics[width=1.0\textwidth]{figures/BSLA_solo/BSLA_solo_rmsf.pdf}  
            \linebreak
            \includegraphics[width=0.65\textwidth]{figures/BSLA_flexibility.png}
        \end{minipage}
    \hspace{0.5cm}
        \begin{minipage}[t]{0.45\linewidth}
            \centering
            \includegraphics[width=1.0\textwidth]{figures/BSLA_solo/BSLA_solo_dist_ASP133_HIS156.pdf} 
            \linebreak
            \includegraphics[width=1.0\textwidth]{figures/BSLA_pocket/BSLA_pocket_cartoon.pdf}
            %\includegraphics[width=0.9\textwidth]{figures/BSLA_solo/BSLA_distribution_ASP133_HIS156.pdf}  
        \end{minipage}
    \end{figure}  

\end{frame}    

% ============================================================================ %

\begin{frame}
    \frametitle{Results}
    \framesubtitle{GMIN Basin Hopping Structures} 

    \vspace{0.10\topmargin}

%    \begin{columns}[t]
%        \column{.5\linewidth}
%        \centering
%        Dimensionality Reduction
%        \begin{figure}
%            \includegraphics[width=1.0\textwidth]{figures/CitA_phi.png}
%        \end{figure}      
%
%        \column{.5\linewidth}
%        \centering
%        Basin Hopping Structures
%        \begin{figure}
%            \includegraphics[width=1.0\textwidth]{figures/GMIN/CitA_phi_theta_3D.pdf}
%        \end{figure}     
%
%    \end{columns}  

        \begin{figure}
            \includegraphics[width=0.8\textwidth]{figures/GMIN/CitA_phi_theta_3D.pdf}
        \end{figure}      


\end{frame}     


% ============================================================================ %

\begin{frame}
    \frametitle{Results}
    \framesubtitle{GMIN Basin Hopping} 

    \vspace{0.06\topmargin}

    \begin{columns}[t]
        \column{.5\linewidth}
        1)
        \vspace{-4ex}
        \begin{figure}
            \includegraphics[width=0.85\textwidth]{figures/Complex_structures/structure1.png}  
        \end{figure}      
        3)
        \vspace{-8ex}
        \begin{figure}
            \includegraphics[width=0.85\textwidth]{figures/Complex_structures/structure3.png}   
        \end{figure}       

        \column{.5\linewidth}
        2)
        \vspace{-4ex}
        \begin{figure}
            \includegraphics[width=0.85\textwidth]{figures/Complex_structures/structure2.png}  
        \end{figure}      
        4)
        \vspace{-8ex}
        \begin{figure}
            \includegraphics[width=0.85\textwidth]{figures/Complex_structures/structure4.png}   
        \end{figure}       

    \end{columns}  


\end{frame}      

% ============================================================================ %

\begin{frame}
    \frametitle{Results}
    \framesubtitle{Fusion Protein MD -- Secondary Structure}  

    \centering
    Secondary Structure is stable during 100 ns \\
    (Supplementary Material)
%    \begin{figure}
%        \includegraphics[width=0.78\textwidth]{figures/DSSP/dssp_presentation.pdf}
%    \end{figure}        
    
\end{frame}      


% ============================================================================ %

\begin{frame}
    \frametitle{Results}
    \framesubtitle{Fusion Protein MD -- Hydrophobic Contacts}  

    \vspace{0.06\topmargin}

    \begin{figure}
        \begin{minipage}[]{0.45\linewidth}
            \centering
            \includegraphics[width=\textwidth]{figures/Complex_hydrophobic_core/hydrophobic_core_linker.png}
        \end{minipage}
    \hspace{0.5cm}
        \begin{minipage}[]{0.45\linewidth}
            \centering
            \includegraphics[width=\textwidth]{figures/Complex_hydrophobic_core/protein.png}
        \end{minipage}
    \end{figure}     
 
    
\end{frame}       

% ============================================================================ %

\begin{frame}
    \frametitle{Results}
    \framesubtitle{Dimensionality Reduction}

    \begin{figure}
        \includegraphics[width=1.0\textwidth]{figures/Collective_coords/collective_coords.pdf}
    \end{figure}        

\end{frame}     
 
% ============================================================================ %

\begin{frame}
    \frametitle{Results}
\framesubtitle{Fusion Protein MD}% -- Trajectory}  

    \vspace{0.94\topmargin}

    \begin{figure}
        \hspace{0.0\textwidth}
        \includegraphics[width=0.40\textwidth]{figures/Collective_coords/collective_coords.pdf}
    \end{figure}     

    \vspace{-0.5cm}
 
%    \vspace{0.06\topmargin}

    \begin{columns}[t]
        \column{.5\linewidth}
        \vspace{-4ex}
        \begin{figure}
            \includegraphics[width=0.85\textwidth]{figures/Complex_trajectory/collecitve_coords_structure1.pdf}  
        \end{figure}      
        \vspace{-5ex}
        \begin{figure}
            \includegraphics[width=0.85\textwidth]{figures/Complex_trajectory/collecitve_coords_structure3.pdf}  
        \end{figure}       

        \column{.5\linewidth}
        \vspace{-4ex}
        \begin{figure}
            \includegraphics[width=0.85\textwidth]{figures/Complex_trajectory/collecitve_coords_structure2.pdf}  
        \end{figure}      
        \vspace{-5ex}
        \begin{figure}
            \includegraphics[width=0.85\textwidth]{figures/Complex_trajectory/collecitve_coords_structure4.pdf}   
        \end{figure}       

    \end{columns}   
    
\end{frame}        

% ============================================================================ %

\begin{frame}
    \frametitle{Results}
    \framesubtitle{Fusion Protein MD -- TYR/TRP Fluorescence}  

%    \vfill
    \vspace{-4ex}

    \begin{columns}[t]
        \column{.5\linewidth}
        \begin{figure}
            \includegraphics[width=1.0\textwidth]{figures/TyrTrp/TyrTrp_experiment.pdf}
        \end{figure}         

        \centering
        Excitation: 278 nm 

        \begin{center}
        \begin{table}
        \tiny
        \begin{tabular}{l c c} 
                       & Absorbtion & Fluorescence \\
            \hline
            Tyrosine   & 280 nm & 348 nm \\
            Tryptophan & 274 nm & 303 nm \\
        \end{tabular}
        \end{table}
        \end{center}

        \column{.5\linewidth}
        \begin{figure}
            \includegraphics[width=1.0\textwidth]{figures/TyrTrp/average_mindist_TyrTrp.pdf}
        \end{figure}      

        \centering
        Tryptophan only in BSLA

    \end{columns}    

    \vfill

    \vspace{2ex}
    \tiny
    \fullcite{FluorescenceWeb}
 
\end{frame}        

% ============================================================================ %

\begin{frame}
    \frametitle{Results}
    %\framesubtitle{Fusion Protein MD -- Active Site Distances}  
    \framesubtitle{Fusion Protein MD}  

    \vspace{0.94\topmargin}

    \begin{figure}
        \hspace{0.1\textwidth}
        \includegraphics[width=.3\linewidth]{figures/CitA_pocket2.pdf}
        \includegraphics[width=.3\textwidth]{figures/BSLA_pocket/BSLA_pocket_cartoon.pdf}
    \end{figure}     

    \vspace{-0.5cm} 

    \vfill

    \begin{columns}[]
        \column{.5\linewidth}
        \vfill
        \centering
        Open / Closed Conformation
        \begin{figure}
            \includegraphics[width=\textwidth]{figures/CitAP_BSLA_distance/BSLA_CitAP_analyzed_with_average_of_last_50_ns.pdf}  
        \end{figure}         

        \column{.5\linewidth}
        \vfill
        \centering
        Flexibility
        \begin{figure}
            \includegraphics[width=\textwidth]{figures/CitAP_BSLA_distance/BSLA_CitAP_analyzed_with_standard_deviation.pdf}  
        \end{figure}      

    \end{columns}    

    \vfill 

\end{frame}        
 
% ============================================================================ %

\section{Conclusions}

\begin{frame}
    \frametitle{Conclusions}

    \begin{itemize}
        \item<1-> Solo MD simulations gave expected results
        \item<2-> 4 different fusion protein structures
        \item<3-> Secondary structure is stable
        \item<4-> 2 hydrophobic cores
        \item<5-> Systems not equilibrated after 100 ns
        \item<6-> Only structure 4 in accordance with TYR/TRP fluorescence
        \item<7-> Binding pocket dynamics different in fusion protein (more flexible)
        \item<8-> No unique active site distance correlation between domains
    \end{itemize}

\end{frame}        

% ============================================================================ %

\section{Outlook}

\begin{frame}
    \frametitle{Outlook}

    \setstretch{2}

    \begin{itemize}
        \item<1-> Lack of experimental data
        \item<2-> Structural information required
            \begin{itemize}
                \item<2-> Full X-ray/NMR structure
                \item<2-> Information on global shape (SAXS)
                \item<2-> Intermolecular distances (FRET)
            \end{itemize}
        \item<3-> Coarse grained MD for equilibration (MARTINI force field)
    \end{itemize}

    \setstretch{1}

\end{frame}   

% ============================================================================ %

\begin{frame}
    \frametitle{Questions}

    \vfill
    \centering
    \Huge ?
    \vfill


\end{frame}   
 
% ============================================================================ %


\section{Supplementary Material}


\begin{frame}
    \frametitle{Methods}
    \framesubtitle{Forcefield: Amber99sb-ildn-nmr}

    Uncertainty weighted objective function: $\chi^2 = \sum_i(x_i^{Exp} - x_i)^2 / \sigma_i^2$

    \begin{figure}
        \includegraphics[width=\linewidth]{figures/forcefield_performance.png}
    \end{figure}        

    \tiny
    \fullcite{proteinFF}

%    K. A. Beauchamp, Y. Lin, R. Das and V. S. Pande,
%    \href{http://pubs.acs.org/doi/abs/10.1021/ct2007814}
%    {Are Protein Force Fields Getting Better?
%    A Systematic Benchmark on 524 Diverse NMR Measurements},
%    \textit{J. Chem. Theory Comput.},
%    8, 1409--1414, 2012
    

\end{frame}   

% ============================================================================ %

\begin{frame}
    \frametitle{Methods}
    \framesubtitle{Citrate Forcefield Parameterization}

    \begin{figure}
        \includegraphics[width=.7\linewidth]{figures/citrate.png}
    \end{figure}      

    \begin{itemize}
        \item Parameterized with the general AMBER force field (GAFF) from Ambertools using ACPYPE
        \item Partial charges come from
        \begin{itemize}
            \item \textbf{Antechamber} -- AM1-BCC (parameterized fit to \textit{ab initio} calculations)
%            \item YASARA AutoSIMLES Server -- ''improved'' AM1-BCC
%            \item \textit{ab initio}
        \end{itemize}
    \end{itemize}

    \tiny
    \fullcite{ACPYPE}

\end{frame}   

% ============================================================================ %

\begin{frame}
    \frametitle{Results}
    \framesubtitle{Fusion Protein MD -- Secondary Structure}  

    \begin{figure}
        \includegraphics[width=0.78\textwidth]{figures/DSSP/dssp_presentation.pdf}
    \end{figure}        
    
\end{frame}      

% ============================================================================ %

\begin{frame}
    \frametitle{Results}
    \framesubtitle{Fusion Protein MD -- Favourable Contacts}  

    \vspace{0.06\topmargin}

    \begin{columns}[t]
        \column{.5\linewidth}
        \vspace{-4ex}
        \begin{figure}
            \includegraphics[width=0.85\textwidth]{figures/Complex_hydrophobic_core/favourable_cont_structure1.pdf}  
        \end{figure}      
        \vspace{-5ex}
        \begin{figure}
            \includegraphics[width=0.85\textwidth]{figures/Complex_hydrophobic_core/favourable_cont_structure3.pdf}  
        \end{figure}       

        \column{.5\linewidth}
        \vspace{-4ex}
        \begin{figure}
            \includegraphics[width=0.85\textwidth]{figures/Complex_hydrophobic_core/favourable_cont_structure2.pdf}  
        \end{figure}      
        \vspace{-5ex}
        \begin{figure}
            \includegraphics[width=0.85\textwidth]{figures/Complex_hydrophobic_core/favourable_cont_structure4.pdf}  
        \end{figure}       

    \end{columns}   
    
\end{frame}       

% ============================================================================ %

\begin{frame}
    \frametitle{Results}
    \framesubtitle{Fusion Protein MD -- Hydrophobic Contacts}  

    \vspace{0.06\topmargin}

    \begin{columns}[t]
        \column{.5\linewidth}
        \vspace{-4ex}
        \begin{figure}
            \includegraphics[width=0.85\textwidth]{figures/Complex_hydrophobic_core/hydrophobic_cont_structure1.pdf}  
        \end{figure}      
        \vspace{-5ex}
        \begin{figure}
            \includegraphics[width=0.85\textwidth]{figures/Complex_hydrophobic_core/hydrophobic_cont_structure3.pdf}  
        \end{figure}       

        \column{.5\linewidth}
        \vspace{-4ex}
        \begin{figure}
            \includegraphics[width=0.85\textwidth]{figures/Complex_hydrophobic_core/hydrophobic_cont_structure2.pdf}  
        \end{figure}      
        \vspace{-5ex}
        \begin{figure}
            \includegraphics[width=0.85\textwidth]{figures/Complex_hydrophobic_core/hydrophobic_cont_structure4.pdf}  
        \end{figure}       

    \end{columns}   
    
\end{frame}       
 
% ============================================================================ %

\end{document}
